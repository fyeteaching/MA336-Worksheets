% !TeX root = main.tex

\hypertarget{sampling-distributions}{%
\section{Sampling Distributions}\label{sampling-distributions}}

\begin{itemize}
\item
  When using sample statistics to estimate population parameter, there
  will be a chance error
  \[\text{Population Parameter}=\text{Sample Statistic}+\text{Chance Error}.\]
\item
  To understand the chance error, we need to know how sample statistics
  distribute. Consider samples of the same size \(n\) randomly chosen
  from the population with replacement.
\item
  The probability distribution of a sample statistic is called a
  \textbf{sampling distribution}.
\item
  The sampling distribution varies as the sample size changes. In
  general, A larger sample size will result a smaller standard deviation
  of the sampling distribution.
\item
  The standard deviation of a sampling distribution is also called the
  \textbf{standard error}.
\end{itemize}

\hypertarget{central-limit-theorem-for-mean}{%
\subsection{Central Limit Theorem for
Mean}\label{central-limit-theorem-for-mean}}

\begin{theorem}[The Central Limit Theorem]
As the sample size \(n\) increases, the sampling distribution of the
sample mean, from a population with the mean \(\mu\) and the standard
deviation \(\sigma\), will approach to a normal distribution with the
mean \(\mu_{\bar{X}}=\mu\) and the standard deviation
\(\sigma_{\bar{X}}=\dfrac{\sigma}{\sqrt{n}}\).
\end{theorem}

\begin{remark}
  \begin{itemize}
    \item 
  In terms of standardization, the central limit
    theorem says that the random variable
    \(\bar{Z}=\dfrac{\bar{x}-\mu}{\sigma/\sqrt{n}}\) has an approximately
    standard normal distribution.
  \item
    For most distributions (not highly skewed), when sample size \(n>30\),
    the sampling distribution of the sample mean \(\bar{X}\) can be
    approximated reasonably well by a normal distribution. The larger the
    sample size, the better the approximation will be.
  \item
    When the population is normally distributed, the sampling distribution of the sample means will be normally distributed for any sample size.
  \item
    If the population distribution is highly skewed, relying on CLT can be
    risky.
  \end{itemize}
\end{remark} 

See the discussion on intuitive explanation:
\url{https://bit.ly/3dtf0q0}

\begin{example}

Randomly draw samples of size 2 with replacement from the numbers 1, 3,
4.

\begin{enumerate}
\item
  Find the sampling distribution of sample means.
\item
  Find the mean, and standard deviation of the sample means.
\item
  Find the mean, and standard deviation of the population.
\item
  How are the means of the population and the sampling distribution
  related.
\item
  How are the standard deviations of the population and the sampling
  distribution related.
\end{enumerate}

\end{example}

\begin{example}

Suppose the mean length of time that a caller is placed on hold when
telephoning a customer service center is 23.8 seconds, with standard
deviation 4.6 seconds. Find the probability that the mean length of time
on hold in a random sample of 1,000 calls will be within 0.5 second of
the population mean.

\end{example}

\vspace*{6\baselineskip}

\begin{example}

Suppose speeds of vehicles on a particular stretch of roadway are
normally distributed with mean 36.6 mph and standard deviation 1.7 mph.

\begin{enumerate}
\item
  Find the probability that the speed \(X\) of a randomly selected
  vehicle is between 35 and 40 mph.
\item
  Find the probability that the mean speed \(\bar{X}\) of 10 randomly
  selected vehicles is between 35 and 40 mph.
\end{enumerate}

\end{example}

\hypertarget{sampling-distribution-of-a-sample-proportion}{%
\subsection{Sampling Distribution of a Sample
Proportion}\label{sampling-distribution-of-a-sample-proportion}}

The proportion of a specific characteristic in a data set can be viewed
as the mean of the data set by identifying the specific characteristic
with 1 and others with \(0\).

\begin{example}

Consider the following data set

\textbf{1}, 0, \textbf{1}, \textbf{1}, 0, 0, \textbf{1}, 0, \textbf{1},
\textbf{1}

\begin{enumerate}
\item
  What proportion of the numbers are in bold?
\item
  What's the mean of the data set?
\item
  Is there any relation between the proportion and the mean? If so,
  describe it.
\end{enumerate}

\end{example}

\begin{itemize}
  \item 
  In general, if a population consisting of 1s and 0s, then the proportion
  \(p\) of 1s is the same as the mean. The standard deviation is
  \[\sigma=\sqrt{(1-p)^2p+(0-p)^2(1-p)}=\sqrt{p(1-p)}.\]
\end{itemize}

\hypertarget{central-limit-theorem-for-proportion}{%
\subsection{Central Limit Theorem for
Proportion}\label{central-limit-theorem-for-proportion}}

For a sampling distribution of sample proportion, we write \(\hat{P}\)
for the random variable of sample proportions.

\begin{theorem}

\textbf{Central Limit Theorem for Proportion:}

For large samples, the distribution of sample proportions \(\hat{P}\) is
approximately normal, with the mean \(\mu_{\hat{P}}=p\) and standard
deviation \(\sigma_{\hat{P}}=\sqrt{\frac{p(1-p)}{n}}\), where \(p\) is
the population proportion.

\end{theorem}

\begin{itemize}
\item
  As a sample proportion is always between 0 and 1, and 99.7\% of sample
  proportions lie within 3 standard deviation away from the population
  proportion, when using the central limit theorem for proportion, we
  require the sample size \(n\) satisfying the following condition: the
  interval
  \[\left[p-3\sqrt{\frac{p(1-p)}{n}}, p+3\sqrt{\frac{p(1-p)}{n}}\right]\]
  lies wholly in the interval \([0, 1]\).
\item
  In practice, if \(n\) satisfies the following two inequalities:
  \(np\ge 10\) and \(n(1-p)\ge 10\), then we consider \(n\) is large
  enough for assuming that the sampling distribution of the sample
  proportion is approximately normal.
\item
  When the population proportion \(p\) is unknown, to apply the central
  limit theorem for proportion, we require the sample size \(n\)
  satisfying the same conditions with \(p\) replaced by the sample
  proportion \(\hat{p}\). That is, the sample size \(n\) should
  satisfies \(n\hat{p}\ge 10\) and \(n(1-\hat{p})\ge 10\).
\end{itemize}

\begin{example}

Suppose that in a population of voters in a certain region 53\% are in
favor of a particular law. Nine hundred randomly selected voters are
asked if they favor the law.

Find the probability that the sample proportion computed from a random
sample of size 900 will be at least 2\% above true population
proportion.

\end{example}

\vspace*{6\baselineskip}

\begin{example}

Suppose that in 36\% of all car accidents involve injury. Find the
probability that the injury rate in a random sample of 250 car accidents
is between 30\% and 45\%.

\end{example}

\vspace*{6\baselineskip}

\begin{exercise}

An unknown distribution has a mean of 28 and a standard deviation 6.
Samples of size n = 30 are drawn randomly from the population. Find the
probability that the sample mean is between 27 and 30.

\end{exercise}

\vspace*{6\baselineskip}

\begin{exercise}

The numerical population of grade point averages at a college has mean
2.61 and standard deviation 0.5. If a random sample of size 100 is taken
from the population, what is the probability that the sample mean will
be between 2.51 and 2.71?

\end{exercise}

\vspace*{6\baselineskip}

\begin{exercise}

An airline claims that 72\% of all its flights to a certain region
arrive on time. In a random sample of 30 recent arrivals, 19 were on
time. You may assume that the normal distribution applies.

\begin{enumerate}
\item
  Compute the sample proportion.
\item
  Assuming the airline's claim is true, find the probability of a sample
  of size 30 producing a sample proportion so low as was observed in
  this sample.
\end{enumerate}

\end{exercise}

\begin{exercise}

In a mayoral election, based on a poll, a newspaper reported that the
current mayor received 45\% of the vote. If this is true, what is the
probability that a random sample of 100 voters had less than 35\% voting
for the current mayor?

\end{exercise}

\vspace*{6\baselineskip}

\hypertarget{more-practice-on-sampling-distributions}{%
\subsection{More Practice on Sampling
Distributions}\label{more-practice-on-sampling-distributions}}

\begin{exercise}

A population has mean 73.5 and standard deviation 2.5.

\begin{enumerate}
\item
  Find the mean and standard deviation of \(\bar{X}\) for samples of
  size 30.
\item
  Find the probability that the mean of a sample of size 30 will be less
  than 72.
\end{enumerate}

\end{exercise}

\begin{exercise}

A normally distributed population has mean 57.7 and standard deviation
12.1.

\begin{enumerate}
\item
  Find the probability that a single randomly selected element X of the
  population is less than 45.
\item
  Find the mean and standard deviation of \(\bar{X}\) for samples of
  size 16.
\item
  Find the probability that the mean of a sample of size 16 drawn from
  this population is less than 45.
\end{enumerate}

\end{exercise}

\begin{exercise}

Suppose the mean amount of cholesterol in eggs labeled ``large'' is 186
milligrams, with standard deviation 7 milligrams. Find the probability
that the mean amount of cholesterol in a sample of 144 eggs will be
within 2 milligrams of the population mean.

\end{exercise}

\vspace*{6\baselineskip}

\begin{exercise}

Suppose that 8\% of all males suffer some form of color blindness. Find
the probability that in a random sample of 250 men at least 10\% will
suffer some form of color blindness.
  
\end{exercise}

\vspace*{6\baselineskip}

\begin{exercise}

An airline claims that 72\% of all its flights to a certain region
arrive on time. In a random sample of 30 recent arrivals, 19 were on
time. You may assume that the normal distribution applies.

\begin{enumerate}
\item
  Compute the sample proportion.
\item
  Assuming the airline's claim is true, find the probability of a sample
  of size 30 producing a sample proportion so low as was observed in
  this sample.
\end{enumerate}

\end{exercise}

\begin{exercise}

A particular fruit's weights are normally distributed, with a mean of
663 grams and a standard deviation of 38 grams.
If you pick 13 fruits at random, then 9\% of the time, their mean weight
will be greater than how many grams?

\end{exercise}

\vspace*{2.5\baselineskip}

\subsection{Lab: Normal Distributions}

\begin{itemize}
\item
  Let \(X\) be a normal random variable with mean \(\mu\) and standard
  deviation \(\sigma\), that is \(X\sim \mathcal{N}(\mu, \sigma^2)\). In
  Excel, \(P(X<x)\) is given by
  \texttt{NORM.DIST(x,\ mean,\ sd,\ TRUE)}.
\item
  Recall the mean of a data set can obtained by the Excel function
  \texttt{AVERAGE()}.
\item
  Given the population mean \(\mu\) and standard deviation \(\sigma\),
  if the sample size \(n\) is bigger than 30 and the sample mean is
  \(\bar{x}\). The probability of getting another sample of the same
  size but smaller mean can be obtained by the following Excel function:
  \texttt{NORM.DIST(} \(\bar{x},\mu,\sigma\) \texttt{/sqrt(n),TRUE)}.
\end{itemize}

\begin{exercise}

CNNBC recently reported that the mean annual cost of auto insurance is
1035 dollars. Assume the standard deviation is 109 dollars. You will use
a simple random sample of 83 auto insurance policies.

\begin{enumerate}
\item
  Find the probability that a single randomly selected policy has a mean
  value between 1028.6 and 1044.8 dollars.
\item
  Find the probability that a random sample of size \(n=79\) has a mean
  value between 1028.6 and 1044.8 dollars.
\end{enumerate}

\end{exercise}

