% !TeX root = main.tex

\hypertarget{confidence-interval-for-a-proportion}{%
\subsection{Confidence Interval for a
Proportion}}

\begin{itemize}
\item
  Recall that the standard error of sample proportions is
  \(\sigma_{\hat{P}}=\sqrt{\frac{p(1-p)}{n}}\), where \(n\) is the
  sample size and \(p\) is the population proportion. As a consequence,
  when estimating the population proportion \(p\), we only have a point
  estimate \(\hat{p}\) (phat) to use. For the standard error, we use the
  estimation
  \[\sigma_{\hat{p}}\approx\hat{\sigma}_{\hat{p}}=\sqrt{\dfrac{\hat{p}(1-\hat{p})}{n}}.\]

\item
  Based on the central limit theorem, when \(n\) is large enough, at the
  \(100(1-\alpha)\%\) level, the margin of error for \(p\) is defined as
  \[E=z_{\alpha/2}\sqrt{\frac{\hat{p}(1-\hat{p})}{n}}\]
  In Excel,\\
  \begin{fullwidth}\raggedleft
    \colorbox{white}{
      \(z_{\alpha/2}\)=\texttt{NORM.S.INV((1\ +\ confidence\ level)/2)}.
  
      \footnote{\footnotesize
        By the central limit theorem, the random variable \(\hat{p}\) is
      normal distributed. The chance that
      \[p\in \left[\hat{p}-z_{\alpha/2}\sqrt{\frac{\hat{p}(1-\hat{p})}{n}}, \hat{p}+z_{\alpha/2}\sqrt{\frac{\hat{p}(1-\hat{p})}{n}}\right]\]
      is the same as the chance that
      \[\hat{p}\in \left[p-z_{\alpha/2}\sqrt{\frac{p(1-p)}{n}}, p+z_{\alpha/2}\sqrt{\frac{p(1-p)}{n}}\right].\]
      That shows \(z_{\alpha/2}\) satisfying
      \[P(-z_{\alpha/2}<\dfrac{\hat{p}-p}{\sqrt{\frac{p(1-p)}{n}}}<z_{\alpha/2})=1-\alpha.\]
      }
    }
  \end{fullwidth}

  The marginal error can also be obtained by

  \begin{fullwidth}
    \colorbox{white}{
      \texttt{CONFIDENCE.NORM(1-confidence\ level,\ SQRT(phat*(1-phat)/n,\ n)}.
    }
  \end{fullwidth}

\item
  The confidence interval for \(p\) is defined by
  \[[\hat{p}-E,\hat{p}+E]=\left[\hat{p}-z_{\alpha/2}\sqrt{\frac{\hat{p}(1-\hat{p})}{n}}, \hat{p}+z_{\alpha/2}\sqrt{\frac{\hat{p}(1-\hat{p})}{n}}\right],\]
  where
  \href{https://saylordotorg.github.io/text_introductory-statistics/s11-01-large-sample-estimation-of-a-p.html}{the
  critical value \(z_{\alpha/2}\) satisfies that
  \(P(Z< z_{\alpha/2})=1-\alpha/2\)} for the standard normal variable
  \(Z\).
\item
  In practical, the sample size \(n\) is considered large enough if
  \(n\hat{p}\ge 10\) and \(n(1-\hat{p})\ge 10\).
\item
  The above defined confidence interval is known as the
  \href{https://en.wikipedia.org/wiki/Binomial_proportion_confidence_interval}{normal
  approximation (or Wald's) confidence interval}. It is popular in
  introductory statistics books. However, it is unreliable when the
  sample size is small or the sample proportion is close to 0 or 1.
  Indeed, if the sample proportion is 0 or 1, the confidence interval
  defined here will have zero length.
\end{itemize}



\begin{example}

In a random sample of 100 students in college, 65 said that they come to
college by bus.

\begin{enumerate}
\item
  Give a point estimate of the proportion of all students who come to
  college by bus.
\item
  Construct a 99\% confidence interval for that proportion.
\end{enumerate}

\end{example}
\vspace*{4\baselineskip}

\begin{example}
  
  Foothill College's athletic department wants to calculate the proportion
  of students who have attended a women's basketball game at the college.
  They use student email addresses, randomly choose 220 students, and
  email them. Of the 145 who responded, 22 had attended a women's
  basketball game.
  
  Calculate and interpret the approximate 90\% confidence interval for the
  proportion of all Foothill College students who have attended a women's
  basketball game.
  
\end{example}
\vspace*{8\baselineskip}


\subsection{Factors Affect the Width of Confidence
Intervals}

\begin{itemize}
\item
  The width of a confidence interval, equals twice the standard error,
  gives a measure of precision of the estimation.
\item
  Recall, for population proportion and mean,
  \[\text{Marginal Error} = \text{Critical Value}\cdot \frac{\text{(estimated) Population SD}}{\sqrt{\text{Sample Size}}}\]
\item
  The formula tells us the precision of a confidence interval is
  affected by the confidence level, the variability, and the sample
  size.

  \begin{itemize}
  \item
    Larger the confidence levels give larger critical values and errors.
  \item
    Populations (and samples) with more variability gives larger errors.
  \item
    Larger sample sizes give smaller errors.
  \end{itemize}
\end{itemize}

\hypertarget{sample-size-determination}{%
\subsection{Sample Size Determination}\label{sample-size-determination}}

\begin{itemize}
\item
  In practice, we may desire a marginal error of \(E\). With a fixed
  confidence level \(100(1-\alpha)\%\), the larger the sample size the
  smaller the marginal error.
\item
  When estimating population proportion, if we can produce a reasonable
  guess \(\hat{p}\) for population proportion, then an appropriate
  minimum sample size for the study is determined by
  \[n=\left(\frac{z_{\alpha/2}}{{E}}\right)^2\cdot \hat{p}(1-\hat{p}).\]
\item
  When estimating population mean, if we can produce a reasonable guess
  \(\sigma\) for the population standard deviation, then an appropriate
  minimum sample size is given by
  \[n=\left(\dfrac{z_{\alpha/2}\cdot \sigma}{{E}}\right)^2.\]
\end{itemize}

\begin{example}

Suppose you want to estimate the proportion of students at QCC who live
in Queens. By surveying your classmates, you find around 70\% live in
Queens. Use this as a guess to determine how many students would need to
be included in a random sample if you wanted the error of margin for a
95\% confidence interval to be less than or equal to 2\%.

\end{example}
\vspace*{8\baselineskip}

\begin{example}

Find the minimum sample size necessary to construct a 99\% confidence
interval for the population mean with a margin of error \(E =0.2\).
Assume that the estimated population standard deviation is
\(\sigma=1.3\).

\end{example}
\vspace*{8\baselineskip}

\begin{exercise}

Out of 400 people sampled, 92 had kids. Based on this, construct a 90\%
confidence interval for the true population proportion of people with
kids.

\end{exercise}
\vspace*{8\baselineskip}

\begin{exercise}

To understand the reason for returned goods, the manager of a store
examines the records on 40 products that were returned in the last year.
Reasons were coded by 1 for ``defective,'' 2 for ``unsatisfactory,'' and
0 for all other reasons, with the results shown in the table.

\begin{fullwidth}
  \colorbox{white}{
    \parbox{\linewidth}{\raggedleft
      \begin{tabular}{*{20}{c}}
        0 & 0 & 0 & 0 & 2 & 0 & 0 & 0 & 0 & 0 & 2 & 0 & 0 & 0 & 0 & 0 & 0 & 0 & 0 & 0 \\
        0 & 0 & 0 & 1 & 0 & 0 & 0 & 0 & 2 & 0 & 2 & 0 & 0 & 0 & 0 & 0 & 0 & 2 & 0 & 0
      \end{tabular}
    }}
\end{fullwidth}

\begin{enumerate}
\item
  Give a point estimate of the proportion of all returns that are
  because of something wrong with the product, that is, either defective
  or performed unsatisfactorily.
\item
  Construct an 80\% confidence interval for the proportion of all
  returns that are because of something wrong with the product.
\end{enumerate}

\end{exercise}

\begin{exercise}

You want to obtain a sample to estimate a population mean. Based on
previous evidence, you believe the population standard deviation is
approximately $\sigma=41.5$. You would like to be
99\% confident that your esimate is within 0.5 of the true population
mean. How large of a sample size is required?

\end{exercise}
\vspace*{8\baselineskip}

\begin{exercise}

Suppose a mobile phone company wants to determine the current percentage
of customers aged 50+ who use text messaging on their cell phones. How
many customers aged 50+ should the company survey in order to be 90\%
confident that the estimated (sample) proportion is within three
percentage points of the true population proportion of customers aged
50+ who use text messaging on their cell phones.

\end{exercise}
\vspace*{8\baselineskip}

\subsection{Lab: Confidence Interval for Proportion}

\begin{itemize}
\item
  When a cumulative probability \(p=P(Z<z)\) of a standard normal random
  variable \(Z\) is given, we can find \(z\) using
  \texttt{NORM.S.INV(p)}.
\item
  If a sample of size \(n\) has the proportion \(\hat{p}\) and the
  sampling distribution is approximately normal, the marginal error for
  the proportion can be obtained by the Excel function
  \vskip 0.5em
  \begin{fullwidth}
    \colorbox{white}{
      \texttt{CONFIDENCE.NORM(1-confidence\ level,\ SQRT(phat*(1-phat)),\ n)}
    }
  \end{fullwidth}
\end{itemize}

\begin{exercise}
  
  Suppose 250 randomly selected people are surveyed to determine if they own a tablet. Of the 250 surveyed, 98 reported owning a tablet. Using a 95\% confidence level, compute a confidence interval estimate for the true proportion of people who own tablets.

\end{exercise}
\vspace*{8\baselineskip}

\begin{exercise}

A software engineer wishes to estimate, to within 5 seconds, the mean
time that a new application takes to start up, with 95\% confidence.
Estimate the minimum size sample required if the standard deviation of
start up times for similar software is 12 seconds.

\end{exercise}
\vspace*{8\baselineskip}

\begin{exercise}

The administration at a college wishes to estimate, to within two
percentage points, the proportion of all its entering freshmen who
graduate within four years, with 90\% confidence. Estimate the minimum
size sample required.

\end{exercise}
\vspace*{8\baselineskip}

