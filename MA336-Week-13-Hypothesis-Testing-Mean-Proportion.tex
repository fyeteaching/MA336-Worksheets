% !TeX root = main.tex

\hypertarget{hypothesis-testing-for-one-sample}{%
\section{Hypothesis Testing for One
Sample}\label{hypothesis-testing-for-one-sample}}

\hypertarget{hypothesis-testing-procedure}{%
\subsection{Hypothesis Testing
Procedure}\label{hypothesis-testing-procedure}}

\begin{enumerate}[sepno]
\item
  Check if the population distribution is approximately normal or sample size is large enough and determine if a \(Z\)-test
  or \(T\)-test can be performed. For proportion, \(Z\)-test may be
  used. For mean, if \(\sigma\) is known, the \(Z\)-test may be used. If
  \(\sigma\) is unknown, the \(T\)-test may be used.
\item
  State the null and alternative hypothesis. The null hypothesis always
  contains the equal sign (and possibly together with a less than or
  greater than symbol, depending on \(H_a\).)
\item
  Set a significance level \(\alpha\). Commonly used levels are
  \(\alpha=0.01\), \(\alpha=0.05\) and \(\alpha=0.1\).
\item
  Calculate the standardized test statistic: the \(Z\)-test statistic or
  the \(T\)-test statistic.
\item
  Calculate the \(P\)-value according to the type of the test.

  \begin{tabular}{cc}
    \toprule
    Sign in \(H_a\) & Type of Test\\
    \midrule
    \(\ne\) & Two-tailed\\
    \(<\) &  Left-tailed\\
    \(>\) & Right-tailed\\
    \bottomrule
  \end{tabular}
  
\item
  Make a test decision about the null hypothesis \(H_0\). We reject
  \(H_0\) if the \(P\)-value less than the significance level
  \(\alpha\).
\item
  State an overall conclusion.
\end{enumerate}
\vspace*{-0.5\baselineskip}

\begin{example}

Residences on a certain street claim that the mean speed of automobiles
run through the street is greater than the speed limit of 25 miles per
hour. A random sample of 100 automobiles has a mean speed of 26 miles
per hour. Assume the population standard deviation is 4 miles per hour.
Is there enough evidence to support the claim of the residences at the
significance level \(\alpha = 0.05\)?

\end{example}
\vspace*{6\baselineskip}

\begin{example}

A certain manufacturer claims that average numbers of candies in a
certain sized bag that they produce is 20. To test the claims, you
collected a random sample of 10 bags and find the mean is 18 and the
standard deviation is 2.7. Assume the numbers of candies are normally
distributed. At the significance level \(\alpha=0.05\), does your
analysis support the manufacturer's claim?

\end{example}
\vspace*{8\baselineskip}

\begin{example}

An instructor would like to know if the students enrolled in a math
course in the current semester performed better than students in the
last semester. The mean final exam from last semester is 75.5. The final
exam scores of 40 randomly selected 40 students were obtained

\begin{fullwidth}
  \colorbox{white}{
    \parbox{\linewidth}{
      \raggedleft
      \begin{tabular}{*{20}{c}}
        93 & 88 & 69 & 74 & 76 & 81 & 78 & 77 & 74 & 63 & 67 & 81 & 80 & 82 & 68 & 88 & 76 & 69 & 75 & 78\\
        75 & 77 & 94 & 87 & 74 & 88 & 63 & 75 & 94 & 88 & 91 & 77 & 76 & 68 & 80 & 88 & 68 & 83 & 72 & 72
      \end{tabular}
    }
  }
\end{fullwidth}

Do the data provide evidence that the students in this semester
performed significantly better on the final than last semester?

\end{example}
\vspace*{8\baselineskip}

\begin{example}

Suppose you want to determine if a coin is fair. You toss the coin 50
times and observe 16 heads and 34 tails. At the significant level 0.01,
do you think that the coin is fair? If not, does the coin favor the head
or tail?

\end{example}
\vspace*{8\baselineskip}

\begin{example}

Globally the long-term proportion of newborns who are male is 51.46\%. A
researcher believes that the proportion of boys at birth changes under
severe economic conditions. To test this belief randomly selected birth
records of 5,000 babies born during a period of economic recession were
examined. It was found in the sample that 52.55\% of the newborns were
boys. Determine whether there is sufficient evidence, at the 10\% level
of significance, to support the researcher's belief.

\end{example}
\vspace*{8\baselineskip}

\begin{remark}

In some books, the standard error of the sample distribution of sample
proportions assuming that \(p=p_0\) is calculated using the
approximation \[
\sigma_{\hat{p}}=\sqrt{\frac{\hat{p}(1-\hat{p})}{n}}.
\]

An arguable explanation is that using the above value for SE will be
consistent with the approach to a hypothesis testing using a confidence
interval in the case that a two-tailed test is preformed.

\end{remark}

\begin{example}

  The mean work week for engineers in a start-up company is believed to be
  about 60 hours. A newly hired engineer hopes that it's shorter. She asks
  ten engineering friends in start-ups for the lengths of their mean work
  weeks. Based on the results that follow, should she count on the mean
  work week to be shorter than 60 hours?
  
  Data (length of mean work week): 70; 45; 55; 60; 65; 55; 55; 60; 50; 55.
  
\end{example}
\vspace*{8\baselineskip}

\hypertarget{practice}{%
\subsection{Practice}\label{practice}}

\begin{exercise}

A college football coach thought that his players could bench press a
mean weight of 275 pounds. It is known that the standard deviation is 55
pounds. Three of his players thought that the mean weight was more than
that amount. They asked 30 of their teammates for their estimated
maximum lift on the bench press exercise. The mean of their maximum lift
is 286.2.

Conduct a hypothesis test using a 2.5\% level of significance to
determine if the bench press mean is more than 275 pounds.

\end{exercise}
\vspace*{8\baselineskip}

\begin{exercise}

In a college report, it says the mean age of students is 23.4 years old.
An instructor thinks that the mean age is younger than 23.4. He randomly
surveyed 50 students and found that the sample mean is 21.5 and the
standard deviation is 1.9. At the significance level \(\alpha=0.025\),
is there enough evidence to support the instructor's estimation?

\end{exercise}
\vspace*{8\baselineskip}

\begin{exercise}

A teacher believes that 85\% of students in the class will want to go on
a field trip to the local zoo. She performs a hypothesis test to
determine if the percentage is the same or different from 85\%. The
teacher samples 50 students and 39 reply that they would want to go to
the zoo. For a 1\% level of significance, would the data support the
teacher's believe?

\end{exercise}
\vspace*{8\baselineskip}

% \begin{exercise}

% Suppose a consumer group suspects that the proportion of households that
% have three cell phones is 30\%. A cell phone company has reason to
% believe that the proportion is not 30\%. Before they start a big
% advertising campaign, they conduct a hypothesis test. Their marketing
% people survey 150 households with the result that 43 of the households
% have three cell phones.

% \end{exercise}
% \vspace*{8\baselineskip}

% \begin{exercise}

% The average household size in a certain region several years ago was
% 3.14 persons. A sociologist wishes to test, at the 5\% level of
% significance, whether it is different now. Perform the test using the
% information collected by the sociologist: in a random sample of 75
% households, the average size was 2.98 persons, with sample standard
% deviation 0.82 person.

% \end{exercise}
% \vspace*{8\baselineskip}

% \begin{exercise}

% The mean score on a 25-point placement exam in mathematics used for the
% past two years at a large state university is 14.3. The placement
% coordinator wishes to test whether the mean score on a revised version
% of the exam differs from 14.3. She gives the revised exam to 30 entering
% freshmen early in the summer; the mean score is 14.6 with standard
% deviation 2.4. Perform the test at the 10\% level of significance.

% \end{exercise}
% \vspace*{8\baselineskip}

\begin{exercise}

The average number of days to complete recovery from a particular type
of knee operation is 123.7 days. From his experience a physician
suspects that use of a topical pain medication might be lengthening the
recovery time. He randomly selects the records of seven knee surgery
patients who used the topical medication. The times to total recovery
were:

128, 135, 121, 142, 126, 151, 123

Assuming a normal distribution of recovery times, perform the relevant
test of hypotheses at the 10\% level of significance.

Would the decision be the same at the 5\% level of significance?

\end{exercise}
\vspace*{8\baselineskip}

\hypertarget{lab-excel-functions-for-hypothesis-testing}{%
\subsection{Lab: Excel Functions for Hypothesis
Testing}\label{lab-excel-functions-for-hypothesis-testing}}

\hypertarget{excel-functions-for-normal-distributions}{%
\subsubsection{Excel Functions for Normal
Distributions}\label{excel-functions-for-normal-distributions}}

\begin{itemize}
\item
  Let \(Z\) be a standard normal random varaible. In Excel, \(P(Z<z)\)
  is given by \texttt{NORM.S.DIST(z,TRUE)}.
\item
  Let \(X\) be a normal random variable with mean \(\mu\) and standard
  deviation \(\sigma\), that is \(X\sim \mathcal{N}(\mu, \sigma^2)\). In
  Excel, \(P(X<x)\) is given by \texttt{NORM.DIST(x,mean,sd,TRUE)}.
\item
  When a cumulative probability \(p=P(X<x)\) of a normal random variable
  \(X\) is given, we can find \(x\) using\\ \texttt{NORM.INV(p,mean,sd)}.
\item
  When a cumulative probability \(p=P(Z<z)\) of a standard normal random
  variable \(Z\) is given, we can find \(z\) using
  \texttt{NORM.S.INV(p)}.
\end{itemize}

\hypertarget{excel-functions-for-t-distributions}{%
\subsubsection{\texorpdfstring{Excel Functions for
\(T\)-Distributions}{Excel Functions for T-Distributions}}\label{excel-functions-for-t-distributions}}

Suppose a Student's \(T\)-distribution has the degree of freedom
\(\text{df}=n-1\).

How to find a probability for a given \(T\)-value?

\begin{itemize}
  \item
    The area of the left tail of the \(T\)-value may be calculated by
    the function \texttt{T.DIST(t,df,true)}.
  \item
    The area of the right tail of the \(T\)-value may be calculated by
    the function \texttt{T.DIST.RT(t,\ df)}.
  \item
    The area of two tails of the \(T\)-value
    (\(t > 0\)) may be calculated by
    function \texttt{T.DIST.2T(t,df)}.
  \end{itemize}
  %   To find the critical value for a given probability \(p\)
  
  %   \begin{itemize}
  %   \item
  %     When the area of the left tail is given, the function\\
  %     \texttt{T.INV(p,df)} may be used.
  %   \item
  %     When the area of both tails is given, the function\\
  %     \texttt{T.INV.2T(p,df)} may be used. This function is good for
  %     construction confidence interval.
  %   \end{itemize}

\begin{exercise}

The average number of days to complete recovery from a particular type
of knee operation is 123.7 days. From his experience a physician
suspects that use of a topical pain medication might be lengthening the
recovery time. He randomly selects the records of seven knee surgery
patients who used the topical medication. The times to total recovery
were:

128, 135, 121, 142, 126, 151, 123

Assuming a normal distribution of recovery times, perform the relevant
test of hypotheses at the 10\% level of significance.

Would the decision be the same at the 5\% level of significance?

\end{exercise}


